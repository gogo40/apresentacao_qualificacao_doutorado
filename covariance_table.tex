\section{Covariance table generation}
\begin{frame}{Covariance table generation}
The covariance table $C$ is a data structure used to model the covariances on a grid. Each entry $C(hx,hy,hz)$ gives the covariance between points $p_1 = (x_1, y_1, z_1)$ and $p_2 = (x_2, y_2, z_2)$ apart of $(hx, hy, hz)$, where $hx = x_2 - x_1$, $hy = y_2 - y_1$ and $hz = z_2 - z_1$.

There are two main ways to generate this table:
\begin{itemize}
	\item Implicit modeling: The data is used directly to generate the table. A smoothing strategy is needed to eliminate numeric artifacts and generate a positive-definite model\cite{cov.paper}.
    \item Explicit modeling: This is the traditional approach. An explicit model using admissible functions is used to fit a curve in an experimental variogram generated from data. From this model the covariance table can be built for every lag at any direction.
\end{itemize}
\end{frame}

\begin{frame}{The relationship between Covariance Table and Density Spectrum}
The density spectrum $s(\omega)$ is the Fourier Transform of the covariance table. As the covariance table is an even function, i. e., $C(h=(hx,hy,hz))=C(-h=(-hx,-hy,-hz))$, then the density spectrum is real and even.

$$
s(\omega)=\mathscr{F}\left[ C(h) \right]
$$
\end{frame}

\begin{frame}{The Bochner's Theorem}
An important result of the Theory of Fourier Transform is the Bochner's Theorem \cite{bochner1939additive}:

\begin{block}{}
A function $f(y)$ is positive-definite if it is the Fourier Transform of a real function $g(y) \geq 0$.
\end{block}
\end{frame}

\begin{frame}{Generating covariance tables using the Bochner's Theorem}
The Bochner's Theorem is an important tool to generate admissible covariance table $C(h)$ from data, as the density spectrum of a covariance table is a real function $s(\omega) \geq 0$.

Before generating the admissible covariance table $C(h)$ is needed to build an experimental covariance table from data. This task can be done using the Marcotte approach \cite{fast.var.paper}, where Fast Fourier Transform (FFT) \cite{johnson2008implementing} is used to fast generate of an experimental covariance table. Or, the experimental covariance table can be generated using a geometric algorithm based on angular and bandwidth tolerances. 
\end{frame}


\begin{frame}{Generating covariance tables using the Bochner's Theorem}

\begin{itemize}
\item After generating the experimental covariance table $\tilde{C}(h)$,  some information may be missing at some lags $h$ or there are many artifacts. In this case, it is needed to fill the missing data and remove the artifacts. An interpolation and smoothing algorithm is needed to adjust the experimental model $\tilde{C}(h)$ and generate an admissible covariance table $C(h)$. 

\item Yao \cite{cov.paper} presents an algorithm to execute the model interpolation and spectrum smoothing.
\end{itemize}

\end{frame}


\begin{frame}{The Yao's algorithm}

\begin{enumerate}
\item Let $\tilde{C}(h)$ be an interpolated experimental covariance model generated using the data with a radial inverse distance interpolation, for example. This will be the first step to smooth the covariance model and remove artifacts. However, it is not assured that model is positive-definite.
\item $\tilde{C}(h)$ is even and real. Then the Bochner's Theorem can be used.
\item Let $\tilde{s}(\omega)$ be the density spectrum of $\tilde{C}(h)$. To generate a positive-definite function $C(h)$ is enough to smooth $\tilde{s}(\omega)$ using a moving average in each point of the density spectrum $\tilde{s}(\omega)$. 
\item Let $s(\omega)$ be the smoothed density spectrum generated in the previous step. Applying the Inverse Fourier Transform $\mathscr{F}^{-1}$ in $s$, an admissible covariance table $C(h) = \mathscr{F}^{-1}[s(\omega)]$ is generated.
\end{enumerate}

\end{frame}

\begin{frame}{A covariance table figure}
A smoothed covariance table generated using the Walker Lake data-set and Yao's algorithm.
\begin{figure}[!ht]
  %\caption{A smoothed covariance table generated using the Walker Lake data-set and Yao's algorithm.}
  \centering
    \includegraphics[height=0.4\textheight, width=0.6\textwidth]{figs/cov_table_fig.png}
    \label{cov_table_ex.fig}
\end{figure}

\end{frame}

\begin{frame}{A density spectrum figure}
A smoothed density spectrum generated using the Walker Lake data-set and the Yao's algorithm.
\begin{figure}[!ht]
  %\caption{A smoothed density spectrum generated using the Walker Lake data-set and the Yao's algorithm.}
  \centering
    \includegraphics[height=0.5\textheight]{figs/dens_spec.png}
    \label{dens_spec.fig}
\end{figure}
\end{frame}


\begin{frame}{Why covariance tables are useful?}
\begin{itemize}
\item The covariance table $C(h)$  can be used in traditional simulation/estimation algorithms as Sequential Gaussian Simulation, Kriging, etc.
\item It can be used in the conditioning step of spectral simulations such as Turning Bands and Fourier Integral Method.
\item The covariance modeling is the most time consuming stage of most geostatistical workflows. A semi-automatic covariance table modeling tool could reduce this time. Mainly, when a multi-variate model based on MLC  is being built \cite{cov.paper}.
\end{itemize}
\end{frame}

\begin{frame}{The covariance table modeling challenges}

\begin{itemize}
\item Yao's algorithm is powerful. But, it is highly dependent of the quality of input data. It demands a good sampling of simulation/estimation domain.

\item To deal with this limitation an explicit model is needed or an auxiliary  model containing the expected covariance would be required.

\item How this auxiliary model can be generated? Derived from a trend map obtained by Radial Base Function or some machine learning estimation technique?

\item This is an open question to be answered.
\end{itemize}

\end{frame}


