\section{Goals}
\begin{frame}{Goals}

The main goal of this work will be investigate the possibility of building a new generation of geostatistical software that uses efficiently the resources of a modern computational infrastructure (like GPUs, clusters, multi-core processors, parallelization technologies like openMP, GRPC, arrayfire, and so on).

To achieve this goal it will be explored the following geostatistical algorithms:

\begin{itemize}
	\item Spectral simulation: Fourier Integral Method (FIM) and Turning Bands (TBSIM).
    \item Covariance table generation;
\end{itemize}
\end{frame}


\section{Methodology}
\begin{frame}{Methodology}
	\begin{itemize}
		\item The \textit{SGeMS} \cite{remy2009applied} will be used as software platform to the algorithm development, because it has many fundamental building blocks for geostatistical algorithms.
        \item A distributed version of the Fourier Integral Method will be developed and applied in real world scenarios. The algorithm limitations and challenges will be explored.
        \item Methodologies to automatic or semi-automatic covariance table generation will be studied.
        \item A new version of \textit{SGeMS}, called \textit{Ar2GeMS HPC}, will be developed to allow the development of distributed version of geostatistical algorithms (Kriging, SGSIM, TBSIM, and so on). This software will simplify the creation of plugins that need to use a set of different computers to execute a complex task.
	\end{itemize}
\end{frame}

\section{Contributions}
\begin{frame}{Contributions}
	In the end of this work, it will delivered:
    \begin{itemize}
    	\item A new package of HPC implementations of several fundamental geostatistical algorithms;
        \item An API and a distributed computational infrastructure to execute geostatistical workflows in a very large dataset;
        \item A new software tool to simplify the development of HPC geostatistical algorithms. 
    \end{itemize}
\end{frame}




