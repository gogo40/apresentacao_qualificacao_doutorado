\section{Goals}
\begin{frame}{Goals}

The main goal of this is to investigate the possibility of building a new generation of geostatistical software that uses efficiently the resources of a modern computational infrastructure (like GPUs, clusters, multi-core processors and HPC libraries).

To accomplish the thesis goal the following geostatistical algorithms will be explored:

\begin{itemize}
	\item Spectral simulation: Fourier Integral Method (FIM) and Turning Bands (TBSIM).
    \item Covariance table generation;
\end{itemize}
\end{frame}


\section{Methodology}
\begin{frame}{Methodology}
	\begin{itemize}
		\item The \textit{SGeMS} \cite{remy2009applied} will be used as software platform to the algorithm development, once it has many fundamental building blocks for geostatistical algorithms.
        \item A distributed version of the Fourier Integral Method will be developed and applied in practical situations. The algorithm limitations and challenges will be explored.
        \item Methodologies to generate automatically or semi-automatically covariance table will be investigated.
	\end{itemize}
\end{frame}

\section{Contributions}
\begin{frame}{Contributions}
	\begin{itemize}
        \item New methodologies to generate automatically or semi-automatically covariance table and density spectrum.
    	\item New parallelization strategies to implement distributed version of spectral simulation algorithms using efficiently the resources available in a modern computational infrastructure.
        \item New architectures and methodologies to develop modern geostatistical software. 
    \end{itemize}
\end{frame}

\section{Outcomes}
\begin{frame}{Outcomes}
	At the end of this work, it will delivered:
    \begin{itemize}
        \item A new version of \textit{SGeMS}, called \textit{Ar2GeMS HPC} to provide distributed version of geostatistical algorithms (Kriging, SGSIM, TBSIM, and so on). This software will simplify the creation of plugins that need to use a set of different computers to execute a complex task.
    	\item A new package of HPC implementations of a variety of fundamental geostatistical algorithms;
        \item An API and a distributed computational infrastructure to execute geostatistical workflows on a very large dataset;
        \item A new software tool to simplify the development of HPC geostatistical algorithms. 
    \end{itemize}
\end{frame}




